\documentclass[12pt,dvipsnames,svgnames,x11names]{article}
\usepackage{../mit}

\allowdisplaybreaks
%
\thispagestyle{empty}%
\pagestyle{plain}%
%
\begin{document}%
\pagecolor{gray!50}
\begin{center}
  \begin{center}
  \vspace*{\fill}
  \textsc{\LARGE Merge Sort}
  \par\bigskip
  \textsc{By:}
  \par\bigskip
  \textsc{\LARGE Dustin Smith}
  \vspace*{\fill}
\end{center}
\end{center}

\newpage

Merge sort is an efficient, general purpose comparison based sorting algorithm. Merge sort is a divide
and conquer algorithm.
\begin{enumerate}
	\item Divide the unsorted list into \(n\) sublists each containing one element.
	\item Repeatedly merge sublists to produce new sorted sublists until there is on sorted sublist.
\end{enumerate}
What would be the time complexity of this algorithm? Similar to a binary search, we are splitting the 
array in half by the mid point. However, unlike a binary search, merge sort has us operating on both splits.
That is, we have \(T(n) = 2\cdot T(n / 2)\) for the sort plus the \(T(n) = 2\cdot n\) for the merge operation. 
The weak version of the Master Theorem is
\[
	T(n) =
	\begin{cases}
		a\cdot T(n / b) + n^c, & n > 1\\
		d, & n = 1
	\end{cases} \rightarrow
	T(n) = 
	\begin{cases}
		\theta(n^c), & \log_b a < c\\
		\theta(n^c\log_b n), & \log_b a = c\\
		\theta(n^{\log_b a}), & \log_b a > c
	\end{cases}
\]
We have that \(a = 2\), \(b = 2\), and \(c = 1\). Therefore, the time complexity is \(\theta(n\cdot\log_2 n)\).
We need \(\theta(n)\) space for the subarrays and \(\theta(n)\) auxiliary space for the initial array.
\begin{python}
def merge_sort(arr: List[int]) -> List[int]:
  n = len(arr)
  if n == 1:
    return arr
    
  mid = arr // 2
  
  left = merge_sort(arr[:mid])
  right = merge_sort(arr[mid:])
  
  return merge(left, right)
  
  
def merge(left: List[int], right: List[int]) -> List[int]:
  final = []
  i, j = 0, 0
  
  while i < len(left) and j < len(right):
    if left[i] < right[j]:
      final.append(left[i])
      i += 1
    else:
      final.append(right[j])
      j += 1
      
  final.extend(left[i:])
  final.extend(right[j:])
  
  return final
\end{python}
\end{document}



