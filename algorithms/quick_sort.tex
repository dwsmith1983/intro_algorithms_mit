\documentclass[12pt,dvipsnames,svgnames,x11names]{article}
\usepackage{../mit}

\allowdisplaybreaks
%
\thispagestyle{empty}%
\pagestyle{plain}%
%
\begin{document}%
\pagecolor{gray!50}
\begin{center}
  \begin{center}
  \vspace*{\fill}
  \textsc{\LARGE Merge Sort}
  \par\bigskip
  \textsc{By:}
  \par\bigskip
  \textsc{\LARGE Dustin Smith}
  \vspace*{\fill}
\end{center}
\end{center}

\newpage

Quick sort is another in place sorting algorithm. An efficient quick sort implementation is somewhat 
faster than merge sort and two to three times faster than heap sort. Quick sort follows the divide and
conquer paradigm.
\begin{enumerate}
	\item If the range has less than two elements, return.
	\item If not, pick a pivot value--depends on the partitioning routine; could be random.
	\item Partition the range; reorder while determining a point of division so that all elements with values
	less than the pivot come before the division and greater after.
	\item Recursively apply.
\end{enumerate}
Two common partitioning schemes are Lomuto and Hoare. However, Lomuto performs poorly on arrays
with many repeated values giving \(\theta(n^2)\) complexity on an array of all repeats. By the Master 
Theorem, we have complexity of \(\theta(n\log_2 n)\) with worst case at \(\theta (n^2)\). The Lomuto 
partitioning scheme is frequently used due to being easier to understand. However, the Hoare partitioning
scheme is more efficient than Lomuto since it does three times fewer swaps.
\begin{python}
# Implementation from The Algorithms Python
def quick_sort(arr: List[int]) -> List[int]:
  if len(arr) < 2:
    return arr
    
  pivot = arr.pop()
  higher = []
  lower = []
  
  for i in arr:
    (higher if i > pivot else lower).append(i)
    
  return quick_sort(lower) + [pivot] + quick_sort(higher)
  
  
# Lomuto
def lomuto_partition(arr: List[int], low_idx: int, high_idx: int) -> int:
  pivot = arr[high_idx]
  i = low_idx - 1
  
  for j in range(low_idx, high_idx):
    if arr[j] <= pivot:
      i += 1
      arr[i], arr[j] = arr[j], arr[i]
  arr[i + 1], arr[high] = arr[high], arr[i + 1]
  return i + 1
  
  
def quick_sort_lomuto(arr: List[int], low: int, high: int) -> None:
  if low < high:
    part_idx = lomuto_partition(arr, low, high)
    
    quick_sort_lomuto(arr, low, part_idx - 1)
    quick_sort_lomuto(arr, part_idx + 1, high)
    
    
# Hoare
def hoare_partition(arr: List[int], low_idx: int, high_idx: int) -> int:
  pivot = arr[low_idx]
  i = low_idx - 1
  j = high_idx + 1
  
  while True:
    i += 1
    
    while arr[i] < pivot:
      i += 1
      
    j -= 1
    
    while arr[j] > pivot:
      j -= 1
      
    if i >= j:
      return j
      
    arr[i], arr[j] = arr[j], arr[i]
    
    
def quick_sort_hoare(arr: List[int], low: int, high: int) -> None:
  if low < high:
    part_idx = hoare_partition(arr, low, high)
    
    quick_sort_hoare(arr, low, part_idx)
    quick_sort_hoare(arr, part_idx + 1, high)
\end{python}

\end{document}



