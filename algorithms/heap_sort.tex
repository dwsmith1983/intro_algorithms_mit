\documentclass[12pt,dvipsnames,svgnames,x11names]{article}
\usepackage{../mit}

\allowdisplaybreaks
%
\thispagestyle{empty}%
\pagestyle{plain}%
%
\begin{document}%
\pagecolor{gray!50}
\begin{center}
  \begin{center}
  \vspace*{\fill}
  \textsc{\LARGE Heap Sort}
  \par\bigskip
  \textsc{By:}
  \par\bigskip
  \textsc{\LARGE Dustin Smith}
  \vspace*{\fill}
\end{center}
\end{center}

\newpage

Heap sort is a comparison based sorting algorithm. With heap sort, we divide the array into sorted and
unsorted regions. We then iteratively shrink the unsorted region by extracting the largest element.
Heap sort is an in place algorithm so the auxiliary space is \(\theta(1)\).
\begin{enumerate}
	\item Build max heap; building the max heap takes \(\theta(n)\) operations.
	\item Swap the first element of the list with the final element. Decrease the range by one.
	\item Sift the new first element to its appropriate index in the heap.
	\item Return to step two.
\end{enumerate} 
The sifting operation is \(\theta(\log_2 n)\) but \(n\) times so \(\theta(n + n\cdot\log_2 n)\) time complexity.
This is simply \(\theta(n\cdot\log_2 n)\).
\begin{python}
def heap_sort(arr: List[int]) -> None:
  if len(arr) == 1:
    return
    
  heapify(arr, len(arr))
  
  end = len(arr) - 1
  
  while end > 0:
    arr[end], arr[0] = a[0], arr[end]
    end -= 1
    sift_down(arr, 0, end)
    
    
def heapify(arr: List[int], len_arr: int) -> None:
  start = (len_arr - 2) // 2
  
  while start > 0:
    sift_down(arr, start, len_arr - 1)
    start -= 1
    
    
def sift_down(arr: List[int], start: int, end: int) -> None:
  root = start
  
  while (root * 2 + 1) <= end:
    child = root * 2 + 1
    swap = root
    
    if arr[swap] < arr[child]:
      swap = child
    if (child + 1) <= end and arr[swap] < arr[child + 1]:
      swap = child + 1
    if swap != root:
      arr[root], arr[swap] = arr[swap], arr[root]
      root = swap
    else:
      return
\end{python}
\end{document}





