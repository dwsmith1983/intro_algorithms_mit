\documentclass[12pt,dvipsnames,svgnames,x11names]{article}
\usepackage{../mit}

\allowdisplaybreaks
%
\thispagestyle{empty}%
\pagestyle{plain}%
%
\begin{document}%
\pagecolor{gray!50}
\begin{center}
  \begin{center}
  \vspace*{\fill}
  \textsc{\LARGE Binary Search}
  \par\bigskip
  \textsc{By:}
  \par\bigskip
  \textsc{\LARGE Dustin Smith}
  \vspace*{\fill}
\end{center}
\end{center}

\newpage

Binary search is a useful algorithm for sorted arrays. Given some array and target value, we would like
to find the index of the target, if it exists. Otherwise, we can return \(-1\).
\begin{table}[h]
	\centering
	\begin{tabular}{| *{16}{|c|} |}
		\hline
			1 & 3 & 4 & 7 & 8 & 10 & 13 & 14 & 18 & 19 & 21 & 24 & 37 & 40 & 45 & 71\\
		\hline
	\end{tabular}
	\caption{Sorted array for binary search.}
	\label{binary_search}
\end{table}
We can approach the binary search algorithm in two ways
\begin{enumerate}
	\item Recursive
	\item Iterative
\end{enumerate}
First, we will consider the recursive case. Given the array in \cref{binary_search} and a target value
of \(7\), what would be our procedure?
\begin{enumerate}
	\item Set start equal to \(0\) and end equal to the length of the array.
	\item If start is greater than end, return \(-1\).
	\item Set the mid to the floor of start plus end divide by two.
	\item If the array value at the mid point is our target value, return the mid point.
	\item If the target is less than the value of the array at the mid point, we will call our  function
	with parameters the array, target, start, and mid point minus \(1\)  repeating the steps.
	\item If not, we call our function with parameters the array, target, mid point plus \(1\), and end.
\end{enumerate}
For the iterative procedure, we can just use a while the start index is less than or equal to the end
incrementing start and decrementing end while we run the checks.

What is the space and time complexity of binary search? The space complexity is relatively easy to see. 
We are only keeping track of three pointers: start, end, and mid. That is, the space complexity of binary
search is \(\theta(1)\), constant space. For the time complexity, how do we step through process? First, 
we split the array so we have \(n / 2\) where \(n\) is the size of the array. The next step we have \(n / 4\)
and so on. For any array, we have \(n / 2^k\) steps.
\[
	\frac{n}{2}, \frac{n}{4}, \frac{n}{8}, \frac{n}{16}, \frac{n}{32}, \ldots\rightarrow \frac{n}{2^k}
\]
We can bound \(n / 2^k\) by \(1\).
\begin{align*}
	n / 2^k & \leq 1\\
	n & \leq 2^k \\
	\log_2 n & \leq k
\end{align*}
Therefore, our worst case time complexity is \(\theta(\log_2 n)\) and best case would be the mid point is
the value \(\theta(1)\) constant time.
\begin{python}
# recursive
def binary_recrusive(
		array: List[int], 
		target: int, 
		start: int, 
		end: int
		) -> int:
  if start > end:
    return -1
		
  mid = (start + end) // 2
	
  if target == array[mid]:
	return mid
		
  if target < array[mid]:
	# search the left half from the next index 
	# before the mid to the start
	return binary_recrusive(array, target, start, mid - 1)
  else:
	# search the right half from the next index 
	# past mid to the end
	return binary_recrusive(array, target, mid + 1, end)
	
	
# iterative
def binary_iterative(array: List[int], target: int) -> int:
  mid, start = 0, 0
  end = len(array) - 1
  
  while start <= end:
    mid = (start + end) // 2
    
    if target == array[mid]:
      return mid
      
    if target < array[mid]:
      end = mid - 1
    else:
      start = mid + 1
  return -1
\end{python}


\end{document}